\chapter{Abstract}


\begin{english}
WebAssembly is a new technology, which is supported in current browsers by now. Since december 2019 WebAssembly is an official Web standard of the W3C. WebAssembly uses its own bytcode, whose execution model is based on a stack machine and promises faster loading times and better performance than JavaScript, for example. Thus it is possible, to bring other programming languages without a transpiler to the Web. However, compilers are necessary, that can produce bytecode for WebAssembly. Java Applets were based on a similar approach by using the Java virtual machine (JVM) within the browser, but this technology is no longer supported in current browsers.

This master's thesis deals with the development of a compiler for MiniJava, a subset of Java. It is shown how features of this programming language will be mapped to WebAssembly. In addition, the runtime system required for execution is described. A special feature is the direct mapping of objects in MiniJava to JavaScript objects. This enables direct DOM access in the browser, for example. The practical use of the created compiler is demonstrated with a browser demo application, which is a Fibonacci calculator. This application was implemented entirely in MiniJava, except for some generic scripts required to load the application.

This master's thesis shows an approach, how new programming languages can be brought to the Web based on WebAssembly.
\end{english}
