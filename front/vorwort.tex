\chapter{Vorwort}
 
Compiler und die Hintergründe von Programmiersprachen faszinieren mich schon seit langer Zeit. Mein Interesse vertiefte sich durch die Programmierlehrverstaltungen im Bachelorstudium und durch eine eigene Lehrveranstaltung für \emph{formale Sprachen, Compiler- und Werkzeugbau} im Masterstudium, sodass für mich eine Masterarbeit in diesem Themenumfeld in Frage gekommen ist. Aus einem Themenvorschlag von Professor Dobler zu \emph{WebAssembly} entstand so die Idee, mich mit dieser vielversprechenden neue Technologie auseinanderzusetzen und einen Compiler für eine Programmiersprache zu entwickeln.

Da ich bisher viel Erfahrung im Java-Umfeld mit Kotlin gesammelt habe, wäre Kotlin als erste Wahl dafür in Frage gekommen. Da die Stärken von Kotlin allerdings eher in der kompakten Ausdrucksweise liegen, die zwar für Kotlin-Programmier vorteilhaft ist, für Compilerbauer allerdings einen Mehraufwand mit sich bringt, fiel die Wahl sehr schnell auf Java. Da Java allerdings (für eine Masterarbeit) immer noch viel zu umfangreich ist, unterstützt mein Compiler nur eine \emph{kleine} Teilmenge der Sprache: \emph{MiniJava}.

Ganz besonders möchte ich mich bei meinem Betreuer FH-Prof. DI Dr. Heinz Dobler bedanken, der mich über den gesamten Entstehungsprozess stets unterstützt hat. Die interessanten Diskussionen und das wertvolle konstruktive Feedback habe ich immer sehr geschätzt! Außerdem möchte ich mich bei meinem Studienkollegen Florian Guld und meiner Schwester Nicole Schöberl bedanken, die meine Arbeit korrekturgelesen haben.

Ich möchte mich besonders auch bei meiner Familie und bei allen bedanken, die mich unterstützt haben.
