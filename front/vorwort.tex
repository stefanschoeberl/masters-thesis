\chapter{Vorwort und Dank}
 
Compiler und die Hintergründe von Programmiersprachen faszinieren mich schon seit langer Zeit. Mein Interesse vertiefte sich durch die Programmierlehranverstaltungen im Bachelorstudium und im Masterstudium durch die Lehrveranstaltung \emph{formale Sprachen, Compiler- und Werkzeugbau}, sodass für mich eine Masterarbeit in diesem Themenumfeld in Frage gekommen ist. Aus einem Vorschlag von Heinz Dobler zu \emph{WebAssembly} entstand so die Idee, mich mit dieser vielversprechenden neuen Technologie auseinanderzusetzen und einen Compiler für eine eigene Programmiersprache zu entwickeln.

Da ich bisher einiges an Erfahrung im Java"=Umfeld mit Kotlin gesammelt habe, wäre es naheliegend gewesen, einen Compiler für Kotlin zu entwickeln. Da die Stärken von Kotlin allerdings eher in der kompakten Ausdrucksweise liegen, die zwar für Kotlin"=ProgrammierInnen vorteilhaft ist, für CompilerbauerInnen allerdings einen Mehraufwand mit sich bringt, fiel die Wahl sehr schnell auf Java. Da Java allerdings (für eine Masterarbeit) immer noch viel zu umfangreich ist, unterstützt mein Compiler nur eine \emph{kleine} Teilmenge der Sprache: \emph{MiniJava}.

Ganz besonders möchte ich mich bei meinem Betreuer Heinz Dobler bedanken, der mich über den gesamten Entstehungsprozess stets unterstützt hat. Die interessanten Diskussionen und das wertvolle konstruktive Feedback habe ich immer sehr geschätzt! Außerdem möchte ich mich bei meiner Schwester Nicole Schöberl und meinem Studienkollegen Florian Guld bedanken, die meine Arbeit Korrektur gelesen haben.

Ich möchte mich besonders auch bei meiner Familie und bei allen bedanken, die mich unterstützt haben.
