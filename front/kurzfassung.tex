\chapter{Kurzfassung}

WebAssembly ist eine neue (2017) Technologie, die mittlerweile in aktuellen Browsern integriert ist. Seit Dezember 2019 ist WebAssembly ein offizieller Web-Standard des W3C. WebAssembly verwendet einen eigenen Bytecode, dessen Ausführungsmodell auf einer Stackmachine basiert und verspricht schnellere Ladezeiten und bessere Performanz als beispielsweise JavaScript. Dadurch ist es möglich, andere Programmiersprachen ohne Transpiler im Web einzusetzen. Dafür sind allerdings Compiler notwendig, die für WebAssembly Bytecode erzeugen können. Java Applets verfolgten einen ähnlichen Ansatz, indem sie die Java Virtual Machine im Browser einsetzten, jedoch wird diese Technologie mittlerweile nicht mehr in aktuellen Browsern unterstützt.

Diese Masterarbeit befasst sich mit der Entwicklung eines Compilers für MiniJava, einer Untermenge der bekannten Programmiersprache Java. Dabei wird gezeigt, wie Sprachkonstrukte der Programmiersprache auf WebAssembly abgebildet werden. Weiters wird auf das zur Ausführung notwendige Laufzeitsystem eingegangen. Eine Besonderheit dabei ist, dass Objekte in MiniJava auf JavaScript-Objekte abgebildet werden, dadurch ist beispielsweise im Browser ein direkter DOM-Zugriff möglich. Die praktische Anwendung des geschaffenen Compilers wird anhand einer Demo-Anwendung im Browser, einem Fibonacci-Rechner, demonstriert. Diese Anwendung wurde, bis auf generische notwendige Skripte zum Laden der Anwendung, zur Gänze in MiniJava implementiert.

Durch diese Masterarbeit soll eine Möglichkeit gezeigt werden, wie neue Programmiersprachen ins Web auf Basis von WebAssembly gebracht werden können.
