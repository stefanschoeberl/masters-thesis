\chapter{Testen des MiniJava-Compilers und Integration in Gradle}
\label{cha:Testen-des-Compilers}

\section{Testen}

Über die gesamte Entwicklungsphase des Compilers muss sichergestellt werden, dass beim Hinzufügen neuer Funktionalitäten, diese wie erwartet funktionieren. Gleichzeitig muss auch gewährleistet werden, dass die bereits bestehende Funktionalität des Compilers nicht (versehentlich) verändert wird.

Grundsätzlich stehen zwei Ansätze Testen zur Verfügung:
\begin{enumerate}
    \item Jede Komponente isoliert testen, dies entspricht klassischen \emph{Unit}-Tests. Hier könnte man testen, ob zum Beispiel die erwünschten Instruktionen für eine Schleife generiert werden. 
    \item Testen, ob ein gegebener MiniJava"=Quelltext nach dem Kompilieren eine erwartete Konsolenausgabe beim Ausführen produziert. Das entspricht eher einem Integrationstest. Dabei werden in jedem Testfall alle Teilprozesse (Kompilieren, Codegenerieren und Ausführen) ausgeführt.
\end{enumerate}

Es wurde der zweite Ansatz gewählt, da er während des Entwickelns eine größere Flexibilität bietet. Die finale Architektur des Compilers entstand erst im Laufe der Entwicklung. Am Anfang war zu erwarten, dass sich interne Strukturen des Compilers laufend ändern könnten. Klassische \emph{Unit}"=Tests müsste man bei größeren Architekturänderungen immer wieder anpassen. Bei Tests, die den gesamten Prozess vom Kompilieren bis zum Ausführen als \emph{Black"=Box} betrachten, ist eine Änderung nicht notwendig, da bei den einzigen zwei Schnittstellen (MiniJava-Quelltext und Konsolenausgabe) zwischen Implementierung und Tests keine Änderungen zu erwarten sind. Außerdem bilden alle Testfälle gemeinsam eine Art Spezifikation für die Programmiersprache.

Als Testframework wird JUnit 5 \cite{JUnit} in Kombination mit AssertJ \cite{AssertJ} eingesetzt. In Listing \ref{lst:simple-test} wird ein Beispiel für einen einfachen Testfall dargestellt.

\pagebreak
\lstinputlisting[label = {lst:simple-test}, caption = {Einfacher Testfall}]{src/testenGradle/simpleTest.kt}

Beim Durchführen des Tests wird im Hintergrund eine Datei (\emph{main.minijava}) mit dem zu testenden MiniJava"=Quelltext angelegt. JUnit 5 bietet die Annotation \lstinline{@TempDir} an, mit der ein temporäres Verzeichnis für jeden Testfall angelegt wird, das nach Ablauf der Tests wieder gelöscht wird \cite{JUnit}. So werden die Testfälle voneinander isoliert.

Hilfsmethoden wie \lstinline{compileAndRunInMainClass} starten den Compiler und führen das erzeugte Modul aus. Am Ende wird die Konsolenausgabe als Zeichenkette zurückgegeben.

Unter Zuhilfenahme von AssertJ lassen sich die erwarteten Ergebnisse bei der Ausführung des Testfalls fast wie ein englischer Satz elegant formulieren, zum Beispiel: \lstinline{assertThat(...).containsExactly(...)} \cite{AssertJ}.

Es ist auch möglich, native Methoden auf diese Art zu testen. In diesem Fall wird zusätzlich ein Stück JavaScript"=Quelltext mitgegeben. Dieser Quelltext wird in die Datei \emph{main.js} gespeichert, die ebenfalls im temporären Verzeichnis liegt. In Listing \ref{lst:native-test} findet sich ein einfaches Beispiel für so einen Testfall.

\lstinputlisting[label = {lst:native-test}, caption = {Testfall für native MiniJava"=Methoden}]{src/testenGradle/nativeTest.kt}

\pagebreak
\section{Gradle-Task und -Plugin}
\label{sec:GradleTask-und-Plugin}

Es ist möglich, den MiniJava"=Compiler direkt, zum Beispiel über die Kommandozeile, aufzurufen. Für praktische Projekte ist allerdings eine Integration in bestehende Build"=Systeme (in diesem Fall Gradle) nützlich, wie es bei anderen Programmiersprachen (zum Beispiel Java und Kotlin) üblich ist.

In der Gradle"=Dokumentation \cite{Gradle} wird beschrieben, wie eigene Erweiterungen für Gradle implementiert werden können. Nachfolgend werden die notwendigen Schritte erläutert, die speziell für den MiniJava"=Compiler notwendig sind, um ihn in Gradle zu integrieren.

Zunächst wird ein eigener \emph{Task} (\lstinline{MiniJavaCompilationTask}) definiert, der das Kompilieren von MiniJava"=Quelltext kapselt. Der Quelltext des \emph{Tasks} ist in Listing \ref{lst:minijava-compilation-task} zu finden. Dieser \emph{Task} ist vom \lstinline{JavaExec}"=\emph{Task} abgeleitet. Der \lstinline{JavaExec}"=\emph{Task} wird von Gradle zur Verfügung gestellt und kann, ähnlich wie die java"=Kommandozeile, kompilierte Bytecode"=Klassen ausführen. Im \lstinline{MiniJavaCompilationTask} wird der \lstinline{JavaExec}"=\emph{Task} konfiguriert, damit er die Hauptklasse des Compilers (\lstinline{MainKt}) mit den gewünschten Parametern aufruft.

\lstinputlisting[label = {lst:minijava-compilation-task}, caption = {Gradle"=\emph{Task} zum Kompilieren von MiniJava"=Quelltexten}]{src/testenGradle/MiniJavaCompilationTask.groovy}

\pagebreak
Der \lstinline{MiniJavaCompilationTask} bietet zusätzlich noch die Methoden \lstinline{inputDirs} und \lstinline{outputDir} an, mit denen der \emph{Task} konfiguriert werden kann. Die Methoden speichern die übergebenen Verzeichnisnamen und leiten daraus die Parameter für den Compiler ab (\lstinline{updateArg}). Über \lstinline{inputs.dir(...)} und \lstinline{outputs.dir(...)} können in Gradle Metainformationen für den \emph{Task} definiert werden, die beschreiben, von welchen Dateien der \emph{Task} abhängig ist und welche Dateien der \emph{Task} produziert. So kann Gradle den Build"=Prozess optimieren, indem beispielsweise nur dann der MiniJava"=Compiler ausgeführt wird, wenn sich eine MiniJava"=Quelltext"=Datei verändert hat.

Der \lstinline{JavaExec}"=\emph{Task} benötigt zum Ausführen einen Klassenpfad, in dem die MiniJava"=Compiler"=Klassen enthalten sind. Dieser Klassenpfad wird über einen eigenen Namen angesprochen (\lstinline{project.configurations.minijava}) und ist unabhängig von anderen, die in Gradle definiert sind, wie beispielsweise \lstinline{implementation}, \lstinline{testImplementation} und \lstinline{compileOnly}.

Der \lstinline{minijava}"=Klassenpfad muss daher zunächst im Gradle"=Projekt definiert werden. Dies ist mit Hilfe eines eigenen Plugins möglich. Der Quelltext des Plugins findet sich in Listing \ref{lst:minijava-gradle-plugin}.

\lstinputlisting[label = {lst:minijava-gradle-plugin}, caption = {Gradle"=Plugin}]{src/testenGradle/MiniJavaPlugin.groovy}

Damit andere Gradle"=Projekte das Plugin finden können, gibt es mehrere Möglichkeiten. Bei der hier eingesetzten Variante genügt es, im Gradle"=Wurzelprojekt ein Verzeichnis \lstinline{buildSrc} zu erstellen, in dem der Quelltext vom \emph{Task} und Plugin gemäß der Verzeichnisstruktur in Listing \ref{lst:buildSrc-structure} abgelegt wird.

\lstinputlisting[label = {lst:buildSrc-structure}, caption = {Aufbau des \emph{buildSrc}"=Moduls}]{src/testenGradle/buildSrc.txt}

Weiters ist darin die Datei \emph{dev.ssch.minijava.properties} notwendig, in welcher der Name der Plugin"=Klasse definiert wird:

\lstinputlisting{src/testenGradle/dev.ssch.minijava.properties}

\lstinline{dev.ssch.minijava} im Dateinamen dieser Datei ist der Bezeichner, unter dem das Plugin von anderen Gradle"=Projekten gefunden werden kann.

\pagebreak
\section{Integration von MiniJava in eine Node.js"=Konsolenanwendung}
\label{sec:NodeJSExample}

Auf Basis der Vorbereitungen im vorherigen Abschnitt kann der Compiler nun in eigenen Projekten eingesetzt werden. Nachfolgend wird die Konfiguration in Gradle, sowie das Einbinden in eine Node.js"=Konsolenanwendung demonstriert. Die Struktur der Node.js"=Konsolenanwendung ist in Listing \ref{lst:demo-nodejs-structure} dargestellt. Die vorgenommene Konfiguration in der Datei \emph{build.gradle} ist in Listing \ref{lst:demo-nodejs-build-gradle} zu finden.

\lstinputlisting[label = {lst:demo-nodejs-structure}, caption = {Aufbau der Node.js"=Konsolenanwendung}]{src/testenGradle/demo-nodejs.txt}

\lstinputlisting[label = {lst:demo-nodejs-build-gradle}, caption = {build.gradle}]{src/testenGradle/build.gradle}

Nachfolgend einige Erklärungen zu \emph{build.gradle}:
\begin{itemize}
    \item In Zeile 5 wird das Plugin mit dem im vorherigen Abschnitt definierten Bezeichner aktiviert.
    \item In Zeile 9 wird der Compiler in den Klassenpfad \lstinline{minijava} gelegt.
    \item In den Zeilen 12 bis 15 wird der vorher definierte \emph{Task} instanziert und mit zwei Quelltext"=Verzeichnissen (Standardbibliothek und das \emph{minijava}"=Verzeichnis) und dem Ausgabe"=Verzeichnis (\emph{build/wasm-module}) parametrisiert.
\end{itemize}

Die Standardbibliothek befindet sich außerhalb des Projektverzeichnisses, um sie in verschiedenen Projekten wiederverwenden zu können. In der Datei \emph{index.js} kann nun das kompilierte MiniJava"=Modul, wie in Node.js üblich, mit \lstinline{require} importiert und anschließend aufgerufen werden. Der Inhalt dieses Skripts ist in Listing \ref{lst:demo-nodejs-index-js} zu finden.

\pagebreak
\lstinputlisting[label = {lst:demo-nodejs-index-js}, caption = {\emph{index.js}}]{src/testenGradle/index.js}

Das Projekt kann nun mit \lstinline{node index.js} gestartet werden.

\vspace{4em}
In diesem Kapitel wurde gezeigt, wie der Compiler getestet werden kann, um seine korrekte Funktionalität sicherzustellen. Weiters wurde gezeigt, wie der Compiler in Gradle integriert werden kann. Die praktische Verwendung wurde anhand einer Node.js"=Konsolenanwendung demonstriert.

Im nächsten Kapitel wird die praktische Verwendung aller bisherigen Bestandteile in einer abgeschlossenen Web"=Anwendung gezeigt.
